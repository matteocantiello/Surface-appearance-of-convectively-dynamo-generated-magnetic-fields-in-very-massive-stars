Several stars with masses greater than $150\mso$ have been found in the R136 region of the Large Magellanic Cloud  \cite{sof_Parker_Goodwin_Kassim_2010}. The evolution of these stars is very uncertain, mostly due to our poor knowledge of mass loss rates for very luminous stars close to the Eddington limit. While the number of stars above $\sim 150\mso$ (Very Massive Stars) is likely very small (certainly is in the local universe), their role is potentially important for astrophysics. Their yields of heavy elements might contribute substantially to the chemical evolution of the universe. Also their final explosions and remnants might be important for the detection of electromagnetic and gravitational signature of distant events. All these phenomena are affected by the rotation rate of the star through internal circulations [...]. Here we explore the main sequence evolution of very massive stars: in particular we discuss the potential role of convectively generated dynamo fields on the angular momentum loss of these stars. We argue that at solar metallicity massive stars above $XXX\mso$ have convectively dynamo generated, strong magnetic fields (of order 1-10kG) at their surface by the end of the main sequence. At slightly higher initial mass such magnetic fields are present for most of the star's H-burning lifetime. This is due to the effect of strong wind mass loss removing mass from the surface and revealing regions of the star previously convective. The average rotation rate of OB stars is found to be around 100-200$\kms$ (typical rotation periods of order 2-3 days) and the core H-burning convective turnover timescale is also of order $\sim$days. Therefore these stars are typically in a regime where we expect $\alpha-\omega$ dynamo to operate (Rossby number is $\le 1$). Simulation of convective core dynamos in this regime show that magnetic fields with amplitude close to equipartition can be sustained. The poloidal component of such fields is then expected to thread the overlying radiative zone of the star. The poloidal component can impact the wind mass loss if reaches the surface.  In such a case the spin-down timescale can be calculated knowing the wind mass loss and the amplitude of the poloidal component of the field. We find this timescale to be shorter than the main sequence timescale in every case (above $XXX\mso$ and at solar metallicity) which implies all very massive stars should lose most of their angular momentum and be slow rotators by the end of core H-burning. At lower metallicity the threshold for magnetized, slow rotator increases toward higher mass. We discuss possible implications for the subsequent evolution of VMS and in particular their death. 