Several stars with masses greater than $150\mso$ have been found in the R136 region of the Large Magellanic Cloud  \cite{sof_Parker_Goodwin_Kassim_2010}. The evolution of these stars is very uncertain, mostly due to our poor knowledge of mass loss rates for very luminous stars close to the Eddington limit. While the number of stars above $\sim 150\mso$ (Very Massive Stars) is likely very small (certainly is in the local universe), their role is potentially very important for astrophysics. Their yields of heavy elements might contribute substantially to the chemical evolution of the universe. Also their final explosions and remnants might be important for the detection of electromagnetic and gravitational signature of distant events. All these phenomena are affected by the rotation rate of the star through internal circulations [...]. Here we explore the main sequence evolution of very massive stars: in particular we discuss the potential role of convectively generated dynamo fields on the angular momentum loss of these stars. We argue that at solar metallicity massive stars above $XXX\mso$ have convectively dynamo generated, strong magnetic fields (of order 1-10kG) at their surface by the end of the main sequence. At slightly higher initial mass such magnetic fields are present for most of the star's H-burning lifetime. This is due to the effect of strong wind mass loss removing mass from the surface and revealing regions of the star previously convective. Since the average rotation rate of OB stars is found to be around 150-200$\kms$, we adopt this value to show that the these stars are all in the regime where the Rossby number is < 1. This favor a convective dynamo in the stellar core (Refs). 